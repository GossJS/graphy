\chapter{Podsumowanie}

Bez wątpienia dziedzina grafów jest bardzo ciekawa i perspektywiczna. Głównie ze względu na mnogość praktycznych zastosowań w świecie rzeczywistym oraz wielką ilość problemów, które rozwiązuje w szerokiej gamie aspektów życia ludzkiego. 

W ramach niniejszej pracy powstała aplikacja \textit{webowa} Graphy służąca do wyświetlania, edycji i przetwarzania grafów. Wymaga ona jeszcze dopracowania: do poprawy jest kilka drobnych błędów, można wprowadzić szereg usprawnień, zrefaktorować kod i dopisać więcej testów automatycznych. Niemniej jednak już teraz może być użyta w~szkołach lub na uczelniach, np. w~celach edukacyjnych. 

Poza małymi poprawkami wspomnianymi wcześniej, jest jeszcze kilka większych funkcjonalności, które warto byłoby wprowadzić, takich jak: stworzenie części serwerowej umożliwiającej logowanie i zapis grafów do bazy danych, udostępnianie grafu innym użytkownikom w czasie rzeczywistym, czy~integracja z grafowymi bazami danych.

Aplikacja Graphy zawiera zbiór funkcjonalności, którego nie oferuje żadna inna istniejąca aplikacja internetowa. Poza tym posiada przyjazny i użyteczny interfejs użytkownika. Kolejnym wyróżnieniem jest obsługa na urządzeniach mobilnych oraz na urządzeniach z ekranem dotykowym. 

Aplikację można rozbudowywać na wiele sposobów. Na stronie demonstracyjnej dodałem statystyki Google Analytics. Mam je zamiar monitorować i jeśli będzie zainteresowanie aplikacją, wówczas dalej ją rozwijać.
