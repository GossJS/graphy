\chapter{Wymagania}

Rozdział ten zawiera wszystkie wymagania funkcjonalne, które powinna spełniać aplikacja, aby praca z grafami była możliwie przystępna.

\section{Tworzenie grafów}
Podstawowym i oczywistym wymaganiem jest, aby użytkownik mógł stworzyć nowy, pusty graf. Ponadto użytkownik powinien mieć możliwość zaimportowania istniejącego grafu oraz wygenerowania znanego grafu, np. cyklu lub grafu pełnego o zadanej ilości wierzchołków. 

\subsection{Importowanie grafów} \label{subsec:import}
Istnieje wiele formatów służących do opisu grafów. Do najpopularniejszych należą \cite{bernard,gephi}

\begin{itemize}
\setlength\itemsep{0em}
\item \texttt{GraphML} -- \textit{Graph Markup Language}
\item \texttt{GEXF} -- \textit{Graph Exchange XML Format}
\item \texttt{JGF} -- \textit{JSON Graph Format}
\item \texttt{DOT} -- format programu Graphviz
\item \texttt{GML} -- \textit{Graph Modeling Language }
\item \texttt{DGML} -- \textit{Directed Graph Markup Language}
\item \texttt{XGMML} -- \textit{eXtensible Graph Markup and Modeling Language}
\end{itemize}

Użytkownik powinien móc wczytać graf w formatach \texttt{GraphML}, \texttt{GEXF} oraz \texttt{JGF}.

\subsubsection{Graph Markup Language (GraphML)}
\begin{listing}[H]
    \caption{Przykład grafu w formacie GraphML}
    \inputminted{xml}{example.graphml}
    \label{lst:graphml-example}
\end{listing}

\subsubsection{Graph Exchange XML Format (GEXF)}
\begin{listing}[H]
    \caption{Przykład grafu w formacie GEXF}
    \inputminted{xml}{example.gexf}
    \label{lst:gexf-example}
\end{listing}

\subsubsection{JSON Graph Format (JGF)} 
\begin{listing}[H]
    \caption{Przykład grafu w formacie JGF}
    \inputminted{json}{example.json}
    \label{lst:jgf-example}
\end{listing}

\subsubsection{DOT Graphviz} 
\begin{listing}[H]
    \caption{Przykład grafu w formacie DOT}
    \inputminted{text}{example.gv}
    \label{lst:dot-example}
\end{listing}

\subsection{Generowanie grafów}

Użytkownik powinien mieć możliwość wygenerowania znanych grafów, dla zadanych parametrów wejściowych:

\begin{itemize}
\setlength\itemsep{0em}
\item Graf pusty
\item Graf liniowy
\item Graf cykliczny
\item Koło
\item Graf pełny (lub turniej)
\item Graf pełny dwudzielny
\item Graf Petersena
\item Drzewa
\end{itemize}

Definicje i przykłady powyższych grafów znajdują się w sekcji \ref{sec:common-graphs}. Ponadto przydatnym dodatkiem w aplikacji będzie możliwość wygenerowania grafu losowego -- o danej ilości wierzchołków oraz parametrem prawdopodobieństwa określającym, czy pomiędzy dwoma wierzchołkami istnieje krawędź.  

\section{Wizualizacja}

oddalanie, przybliżanie

layouts (grid, circle, concentric, bfs)

samodzielne ustawianie wierzchołków i force layout

różne typy wierzchołków / kolory / ikony w wierzchołkach

style / kolor krawędzi

wyszukiwanie po danych

\section{Edycja}

osobny tryb edycji

dodawanie/usuwanie wierzchołków/krawędzi

dodawanie etykiet / własności

grupowanie wierzchołków


\section{Przetwarzanie}

Podstawowe algorytmy:

\begin{itemize}
\setlength\itemsep{0em}
\item Wyszukiwanie najkrótszej ścieżki
\item Minimalne drzewo rozpinające
\item Page rank
\item Spójne składowe
\item Cykl Eulera
\item Cykl Hamiltona
\end{itemize}

\section{Eksportowanie}
Użytkownik powinien mieć możliwość wyeksportowania do formatów, które zostały przedstawione w podsekcji \ref{subsec:import}. Ponadto przydatną funkcjonalnością będzie możliwość wyeksportowania obecnego widoku do pliku graficznego, np. \texttt{PNG} lub \texttt{JPG}. 