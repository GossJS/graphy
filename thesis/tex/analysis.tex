\chapter{Analiza}
\section{Formaty zapisu grafów} \label{sec:graph-formats}
Istnieje wiele formatów służących do opisu grafów. Do najpopularniejszych należą \cite{bernard,gephi}

\begin{itemize}
\setlength\itemsep{0em}
\item \texttt{GraphML} -- \textit{Graph Markup Language}
\item \texttt{GEXF} -- \textit{Graph Exchange XML Format}
\item \texttt{JGF} -- \textit{JSON Graph Format}
\item \texttt{DOT} -- format programu Graphviz
\item \texttt{GML} -- \textit{Graph Modeling Language }
\item \texttt{DGML} -- \textit{Directed Graph Markup Language}
\item \texttt{XGMML} -- \textit{eXtensible Graph Markup and Modeling Language}
\end{itemize}

\subsubsection{Graph Markup Language (GraphML)}
\begin{listing}[H]
    \caption{Przykład grafu w formacie GraphML}
    \inputminted{xml}{example.graphml}
    \label{lst:graphml-example}
\end{listing}

\subsubsection{Graph Exchange XML Format (GEXF)}
\begin{listing}[H]
    \caption{Przykład grafu w formacie GEXF}
    \inputminted{xml}{example.gexf}
    \label{lst:gexf-example}
\end{listing}

\subsubsection{JSON Graph Format (JGF)} 
\begin{listing}[H]
    \caption{Przykład grafu w formacie JGF}
    \inputminted{json}{example.json}
    \label{lst:jgf-example}
\end{listing}

\subsubsection{DOT Graphviz} 
\begin{listing}[H]
    \caption{Przykład grafu w formacie DOT}
    \inputminted{text}{example.gv}
    \label{lst:dot-example}
\end{listing}

\section{Biblioteki do wizualizacji grafów w JavaScript}

\subsection{Cytoscape.js}

Biblioteka z otwartym źródłem (ang. \textit{open-source}) do analizy i wizualizacji grafów. Udostępniona na zasadach licencji MIT. Została napisana w czystym JavaScript i nie posiada zależności do żadnych innych bibliotek. Cytoscape.js jest następcą porzuconego projektu Cytoscape Web korzystającego z technologii Adobe Flash \cite[309]{franz}. 

Prawa własności intelektualnej posiada do niej Cytoscape Consortium -- organizacja non-profit, która promuje rozwój i dystrybucję oprogramowania związanego z sieciami biologicznymi. Cytoscape.js została stworzona na University of Toronto. Jej głównym kontrybutorem jest Max Franz. Biblioteka została sfinansowana przez granty NRNB (\textit{National Resource for Network Biology}) i NIH (\textit{National Institutes of Health}). Kilka innych uniwersytetów oraz firm również pomagało w rozwoju biblioteki \cite{cytoscape}. 

Cytoscape.js jest kompatybilny z kilkoma przydatnymi bibliotekami oraz środowiskami JavaScript, takimi jak: NodeJS, Browserify, Webpack, Require.js czy Bower, co pozwala na integrację z szeroką gamą systemów opartych na JS. 



\subsection{sigma.js}
\subsection{VivaGraph.js}
\subsection{Linkurious.js}

\begin{table}[H]
\begin{tabularx}{\textwidth}{|r|c|c|c}
\hline 
 & Cytoscape.js & Sigma & VivaGraphJS \\ 
\hline 
Licencja & MIT & MIT & BSD 3 \\ 
\hline 
Rozmiar & 294 & 112,9 & 60,4 \\ 
\hline 
Renderowanie & & & \\
SVG & • & tak & • \\
HTML5 Canvas & • & tak & • \\
WebGL Canvas & • & tak & • \\ 
\hline 
Obsługiwane formaty & • & • & • \\ 
\hline
Rozszerzalność & • & • & • \\ 
\hline 
• & • & • & • \\ 
\hline 
\end{tabularx} 
\end{table}


